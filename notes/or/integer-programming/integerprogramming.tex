\documentclass{article}

\usepackage{amsmath}
\usepackage{amssymb}
\usepackage{amsthm}
\usepackage{graphicx}
\usepackage{hyperref}
\usepackage{float}

\title{Integer Programming}
\author{Mehmet Hakan Satman (Ph.D.)}
\date{\today}

\begin{document}
\maketitle
\tableofcontents

\section{The Knapsack Problem}

\begin{table}[H]
\centering
\begin{tabular}{|c|c|c|c|c|c|c|}
\hline
 Items   &  1  & 2  &  3  &  4 & 5 & 6 \\ 
 \hline
Weight  &  2  & 3  & 4  & 5 & 9 & 7 \\ 
\hline
Value   &  3  & 4  & 5  & 6 & 10 & 8 \\ 
\hline
\end{tabular}
\caption{Knapsack problem data}
\end{table}

The capacity of the knapsack is 15 kg. 
We want to maximize the total value of the items in the knapsack without exceeding its capacity.

$$
\begin{aligned}
\text{Maximize } & Z = 3x_1 + 4x_2 + 5x_3 + 6x_4 + 10x_5 + 8x_6 \\
\text{Subject to } & \\
& 2x_1 + 3x_2 + 4x_3 + 5x_4 + 9x_5 + 7x_6 \leq 15 \\
& x_i \in \{0, 1\} \quad \text{for } i = 1, 2, \ldots, 6
\end{aligned}
$$

\noindent Now, suppose that the item 3 and item 4 are related, meaning that if we include item 3 in the knapsack, 
we must also include item 4 (If we include item 4, we must also include item 3). This can be modeled with the following constraint:

$$
\begin{aligned}
\text{Maximize } & Z = 3x_1 + 4x_2 + 5x_3 + 6x_4 + 10x_5 + 8x_6 \\
\text{Subject to } & \\
& 2x_1 + 3x_2 + 4x_3 + 5x_4 + 9x_5 + 7x_6 \leq 15 \\
& x_3 = x_4 \\
& x_i \in \{0, 1\} \quad \text{for } i = 1, 2, \ldots, 6
\end{aligned}
$$

\noindent Here is the Julia solution using Operations Research package:

\begin{verbatim}
using OperationsResearchModels 
values = [3, 4, 5, 6, 10, 8]
weights = [2, 3, 4, 5, 9, 7]
capacity = 15
problem = KnapsackProblem(values, weights, capacity)
result = solve(problem)
\end{verbatim}


\noindent The output: 


\begin{verbatim}
julia> result.objective
18.0

julia> result.selected
6-element Vector{Bool}:
 1
 1
 1
 1
 0
 0

\end{verbatim}



\section{Project Selection Problem}

\begin{table}[H]
\centering
\begin{tabular}{|c|c|c|c|c|}
\hline
Project & Year 1 & Year 2 & Year 3 & Return \\
\hline
  1     &  50    &   20   &   30   &  200   \\
\hline
  2     &  60    &   30   &   20   &  250   \\
\hline
  3     &  40    &   10   &   50   &  150   \\
\hline
  4     &  30    &   40   &   30   &  180   \\
\hline
    5     &  20    &   50   &   40   &  160   \\
\hline
Budget & 150    &  100   &  120   &        \\
\hline
\end{tabular}
\caption{Project selection data}
\end{table}

Project 1 requires 50 units of resource in year 1, 20 units in year 2, and 30 units in year 3.
When the project is completed, it returns 200 units. 
First year bagget is 150 units, second year budget is 100 units, and third year budget is 120 units.
We want to select projects to maximize the total return without exceeding the yearly budgets.


$$
\begin{aligned}
\text{Maximize } & Z = 200x_1 + 250x_2 + 150x_3 + 180x_4 + 160x_5 \\
\text{Subject to } & \\
& 50x_1 + 60x_2 + 40x_3 + 30x_4 + 20x_5 \leq 150 \quad \text{(Year 1 Budget)} \\
& 20x_1 + 30x_2 + 10x_3 + 40x_4 + 50x_5 \leq 100 \quad \text{(Year 2 Budget)} \\
& 30x_1 + 20x_2 + 50x_3 + 30x_4 + 40x_5 \leq 120 \quad \text{(Year 3 Budget)} \\
& x_i \in \{0, 1\} \quad \text{for } i = 1, 2, \ldots, 5
\end{aligned}
$$

\noindent Here is the Julia solution using Operations Research package:

\begin{verbatim}
using JuMP, HiGHS
model = Model(HiGHS.Optimizer)
@variable(model, x[1:5], Bin)
@objective(model, Max, 200x[1] + 250x[2] + 150x[3] + 180x[4] + 160x[5])
@constraint(model, 50x[1] + 60x[2] + 40x[3] + 30x[4] + 20x[5] <= 150)
@constraint(model, 20x[1] + 30x[2] + 10x[3] + 40x[4] + 50x[5] <= 100)
@constraint(model, 30x[1] + 20x[2] + 50x[3] + 30x[4] + 40x[5] <= 120)
optimize!(model)
println(value.(x))
\end{verbatim}

\noindent The output:

\begin{verbatim}
[1.0, 1.0, 0.0, 1.0, 0.0]
\end{verbatim}



\section{The Assignment Problem}

\begin{table}[H]
\centering
\begin{tabular}{|c|c|c|c|c|}
\hline
         & Task 1 & Task 2 & Task 3 & Task 4 \\
\hline
    Worker 1 &   9    &   2    &   7    &   8    \\
\hline
    Worker 2 &   6    &   4    &   3    &   7    \\
\hline
    Worker 3 &   5    &   8    &   1    &   8    \\
\hline
    Worker 4 &   7    &   6    &   9    &   4    \\
\hline
\end{tabular}
\caption{Assignment problem cost matrix}
\end{table}

We have 4 workers and 4 tasks. The cost of assigning each worker to each task is given in the table above.
We want to assign each worker to exactly one task such that the total cost is minimized.

$$
\begin{aligned}
\text{Minimize } & Z = 9x_{11} + 2x_{12} + 7x_{13} + 8x_{14} + 6x_{21} + \ldots + 9x_{43} + 4x_{44} \\
\text{Subject to } & \\
& x_{11} + x_{12} + x_{13} + x_{14} = 1 \quad \text{(Worker 1 assigned to one task)} \\     
& x_{21} + x_{22} + x_{23} + x_{24} = 1 \quad \text{(Worker 2 assigned to one task)} \\
& x_{31} + x_{32} + x_{33} + x_{34} = 1 \quad \text{(Worker 3 assigned to one task)} \\ 
& x_{41} + x_{42} + x_{43} + x_{44} = 1 \quad \text{(Worker 4 assigned to one task)} \\
& x_{11} + x_{21} + x_{31} + x_{41} = 1 \quad \text{(Task 1 assigned to one worker)} \\
& x_{12} + x_{22} + x_{32} + x_{42} = 1 \quad \text{(Task 2 assigned to one worker)} \\
& x_{13} + x_{23} + x_{33} + x_{43} = 1 \quad \text{(Task 3 assigned to one worker)} \\ 
& x_{14} + x_{24} + x_{34} + x_{44} = 1 \quad \text{(Task 4 assigned to one worker)} \\
& x_{ij} \in \{0, 1\} \quad \text{for } i, j = 1, 2, 3, 4
\end{aligned}
$$

\noindent Here is the Julia solution using Operations Research package:

\begin{verbatim}
using OperationsResearchModels
costs = [9 2 7 8; 6 4 3 7; 5 8 1 8; 7 6 9 4]
problem = AssignmentProblem(costs)
result = solve(problem)
\end{verbatim}

\noindent The output:

\begin{verbatim}
julia> result.cost
13.0

julia> result.solution
4×4 Matrix{Float64}:
  0.0  1.0   0.0  0.0
  1.0  0.0  -0.0  0.0
  0.0  0.0   1.0  0.0
 -0.0  0.0   0.0  1.0
\end{verbatim}

\noindent This means:

\begin{itemize}
\item Worker 1 is assigned to Task 2
\item Worker 2 is assigned to Task 1
\item Worker 3 is assigned to Task 3
\item Worker 4 is assigned to Task 4
\end{itemize}

\end{document}